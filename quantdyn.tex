\documentclass[11pt,a4paper]{jsarticle}
	\usepackage{amsmath,amssymb}
	\usepackage{amsfonts}
	\usepackage{bm}
	\usepackage[dvipdfmx]{graphicx}
	\usepackage{ascmac}
	\usepackage{okumacro}
	\usepackage{titlesec}
	\usepackage{here}%画像を強制的に指定場所に置く[Here]を使える
	\usepackage{cases}%連立方程式など
	\setlength{\textwidth}{\fullwidth}
	\setlength{\textheight}{39\baselineskip}
	\addtolength{\textheight}{\topskip}
	\setlength{\voffset}{-0.5in}
	\setlength{\headsep}{0.3in}
	%
	\makeatletter
	\def\section{\@startsection {section}{1}{\z@}{-2.5ex plus -1ex minus -.2ex}{2.5 ex plus .2ex}{\LARGE\bf}}
	\def\subsection{\@startsection {subsection}{1}{\z@}{-1.5ex plus -1ex minus -.2ex}{2.3 ex plus .2ex}{\Large\bf}}
	\def\subsubsection{\@startsection {subsubsection}{1}{\z@}{-2.5ex plus -1ex minus -.2ex}{.3 ex plus .2ex}{\large \bf}}
	\makeatother

	\makeatletter
	    \renewcommand{\theequation}{%
	    \thesection.\arabic{equation}}
	    \@addtoreset{equation}{section}
	  \makeatother


	\newcommand{\divergence}{\mathrm{div}\,}  %ダイバージェンス
	\newcommand{\grad}{\mathrm{grad}\,}  %グラディエント
	\newcommand{\rot}{\mathrm{rot}\,}  %ローテーション
	\newcommand{\pt}{\partial}  %パーシャル
	\newcommand{\df}{\overset{\mathrm{def}}{=}} %def
	\newcommand{\non}{\nonumber}  %\nonumber
	\newcommand{\dis}{\displaystyle}
	\newcommand{\f}[1]{\framebox[1cm]{\textgt{\small #1}}}
	\newcommand{\kine}{\frac{1}{2}mv^2} %運動エネルギー
	\pagestyle{plain}
		%\markright{\footnotesize \sf 物理数学1A 第一回中間テスト(2017) 解答例 \ %左上のタイトル
		%{@sakuPonit}} %名前
\begin{document}
%% 目次
\tableofcontents
\newpage

\section{光の粒子性と電子の波動性 (第一回 2018/9/18)}

	\subsection{Einsteinによる光の粒子性}
		\subsubsection{光子 -photon-}

			振動数$\nu$,波長$\lambda$の電磁波は,
			「エネルギー
			\begin{equation}
				E = h \nu \label{ehnu}
			\end{equation}
			運動量
			\begin{equation}
				 p = \frac{h}{\lambda} \label{phl}
			\end{equation}
			の光子の集合体」
			であるとした.\par
			光子について
			\begin{equation}
				\lambda = \frac{h}{\nu} \label{lhnu}
			\end{equation}
			より(\ref{phl})は,
			\begin{equation}
				p = h\frac{\nu}{c} \label{phnu}
			\end{equation}

			(\ref{ehnu}),(\ref{phnu})より,
			\begin{equation}
				E = {c}{p} ←光子について成り立つ
			\end{equation}
		\subsubsection{(注)相対論}
				総エネルギーの式
				\begin{equation}
					E^2 = {m_0}^2c^4 + p^2 c^2
				\end{equation}

				($m_0$:静止質量,$p$:$mv$)
				運動質量$m$は
				\begin{equation}
					m\sqrt{1 - \frac{v^2}{c^2}} = m_0
				\end{equation}
				光子について
				$v = c$より
				$m_0 = 0$
				よって
				$E = pc$
				で一致している.
		\subsubsection{[問]光子}
				あるレーザーポインター
				(出力5mW 赤色($\lambda$=650nm))\par
					(1)このレーザ=の光子の運動量は? \par
					(2)このレーザーから毎秒何個の光子が出ているか?

	\subsection{物質波 -matter wave-}
		\subsubsection{物質波}
			de Broglieは光の粒子性を逆に読み,物質が波動を伴うとした.
			エネルギー$E$,運動量$p$を持つ粒子は,
			\begin{eqnarray}
				\nu = \frac{E}{h} \\
				\lambda = \frac{h}{p}
			\end{eqnarray}
			の波動を伴う.($\lambda$はde Broglie波長)\par
			粒子について
			\begin{equation}
				\lambda = \frac{h}{p} = \frac{h}{mv}
			\end{equation}
			(光子の場合$\lambda = c/\nu$) \\
			粒子の場合$\lambda = v/\nu$.
			$v$は位相速度.

			\subsubsection{[問] }
				ド・ブロイ波長 0.2nmの中性子の速度vを計算せよ.
				ただし,中性子の質量$ m = 1.675\times10^{27} $kgとする.
		\subsubsection{光電効果-photoelectric effect-}

				\begin{equation}
					\kine = h \nu - W
				\end{equation}
				$\kine$は電子の運動エネルギー,
				$W$は仕事関数.
		\subsubsection{コンプトン効果-Compton effect-}

			\begin{numcases}
				{}
				(mv)^2 = \left(\frac{n}{\lambda} \right) ^2
				+ \left(\frac{n}{\lambda'}\right)^2
				- 2 \frac{n}{\lambda}\frac{n}{\lambda'}\cos \theta & \\
				\kine = \frac{hc}{\lambda} - \frac{hc}{\lambda '} &
			\end{numcases}

			これらから,コンプトン公式
			\begin{equation}
				(\Delta\lambda \df ) \lambda' - \lambda = \frac{h}{mc}(1-\cos\theta)
			\end{equation}
			が導ける.なお$\dis \frac{h}{mc} = 2.4 \times 10^{-12}$mは
			コンプトン波長と呼ばれる.


		\subsubsection{[問]コンプトン効果}
			$\lambda = 0.124$nmのX線を用いてコンプトン散乱実験する.
			散乱されたX線の波長が1%伸びるような散乱角$\theta$ を求めよ.

\section{波動関数と確率密度関数 (第三回 2018/10/02)}
	\subsection{Problem 1.4 / $\Psi(x,t)$が実数関数}

		\begin{itembox}[l]{Problem 1.4}
			At time t = 0 a particle is represented by the wave function
			\begin{equation}
				\Phi(x,t) =
				\begin{cases}
					A \dis{\frac{x}{a}} & \text{if $0 \leq x \leq a$} \\ \\
					A \dis{\frac{(b-x)}{(b-a)}} & \text{if $a \leq x \leq b$} \\ \\
					0 & \text{otherwise}
				\end{cases}
			\end{equation}
			where $A$, $a$, and $b$ are constants.\\
			(a) Normalize $Ψ$ (that is, find $A$, in terms of $a$ and $b$).\\
			(b) Sketch $Ψ(x, 0)$, as a function of $x$.\\
			(c) Where is the particle most likely to be found, at $t = 0$?\\
			(d) What is the probability of finding the particle to the left of $a$? Check your result in the limiting cases $b = a$ and $b = 2a$.\\
			(e) What is the expectation value of $x$?
		\end{itembox}

		[解答] \\
		(a)$\Psi$を規格化せよ.
		Point を使う.

			\begin{eqnarray}
				\rho = \lvert \Psi \rvert^2 =
			\end{eqnarray}
		\\
		(b)


	\subsection{Problem 1.5 / $\Psi(x,t)$が複素関数}



ありがとう!




\end{document}
