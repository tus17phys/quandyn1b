\documentclass[11pt,a4paper]{jsarticle}
	\usepackage{amsmath,amssymb}
	\usepackage{amsfonts}
	\usepackage{bm}
	\usepackage[dvipdfmx]{graphicx}
	\usepackage{ascmac}
	\usepackage{okumacro}
	\usepackage{titlesec}
	\usepackage{here}%画像を強制的に指定場所に置く[Here]を使える
	\usepackage{cases}%連立方程式など
	\usepackage{color}%文字色 \textcolor{文字色}{文字}
	\setlength{\textwidth}{\fullwidth}
	\setlength{\textheight}{39\baselineskip}
	\addtolength{\textheight}{\topskip}
	\setlength{\voffset}{-0.5in}
	\setlength{\headsep}{0.3in}
	%
	\makeatletter
	\def\section{\@startsection {section}{1}{\z@}{-2.5ex plus -1ex minus -.2ex}{2.5 ex plus .2ex}{\LARGE\bf}}
	\def\subsection{\@startsection {subsection}{1}{\z@}{-1.5ex plus -1ex minus -.2ex}{2.3 ex plus .2ex}{\Large\bf}}
	\def\subsubsection{\@startsection {subsubsection}{1}{\z@}{-2.5ex plus -1ex minus -.2ex}{.3 ex plus .2ex}{\large \bf}}
	\makeatother

	\makeatletter
	    \renewcommand{\theequation}{%
	    \thesection.\arabic{equation}}
	    \@addtoreset{equation}{section}
	  \makeatother


	\newcommand{\divergence}{\mathrm{div}\,}  %ダイバージェンス
	\newcommand{\grad}{\mathrm{grad}\,}  %グラディエント
	\newcommand{\rot}{\mathrm{rot}\,}  %ローテーション
	\newcommand{\pt}{\partial}  %パーシャル
	\newcommand{\df}{\overset{\mathrm{def}}{=}} %def
	\newcommand{\non}{\nonumber}  %\nonumber
	\newcommand{\dis}{\displaystyle}
	\newcommand{\f}[1]{\framebox[1cm]{\textgt{\small #1}}}
	\newcommand{\kine}{\frac{1}{2}mv^2} %運動エネルギー
	\newcommand{\average}[1]{\ensuremath{\langle#1\rangle} } %平均値\average{A} -> <A>
	\pagestyle{plain}
		%\markright{\footnotesize \sf 物理数学1A 第一回中間テスト(2017) 解答例 \ %左上のタイトル
		%{@sakuPonit}} %名前
\begin{document}
\section{はじめに}
		このドキュメントでは『物理学Ⅱ』(TUSテキスト)をの内容をテストの範囲内で簡単にまとめる.
		範囲については,およそ「2.1 1次元の弦の波動方程式(p.7) 〜 3.4調和振動子型ポテンシャルに
		束縛された粒子(p.30)」である.各節のまとめは以下の通り.
		\begin{itembox}[l]{2.1節のまとめ 1次元の弦の波動方程式}
			\begin{itemize}
				\item 古典的な弦を伝わる波を記述する波動方程式は,三角関数
							や指数関数で表される進行波解をもつ.
				\item 波数と角周波数の関係を分散関係という.
							進行波解で位相が一定の所の進む速度を,位相速度という.
			\end{itemize}
			\end{itembox}

		\begin{itembox}[l]{2.2節のまとめ シュレーディンガー方程式}
			\begin{itemize}
				\item 量子の振る舞いを記述する波動方程式はシュレーディンガー方程式である:
							$$ i\hbar\frac{\pt}{\pt t}\Psi(x,t) = \left[ -\frac{\hbar ^2}{2m}\frac{\pt ^2}{\pt x^2} + V(x,t)\right]\Psi(x,t) $$
				\item シュレーディンガー方程式は本質的に複素数の解を持ちうる.
				\item シュレーディンガー方程式は線形なので,重ね合わせの原理が成り立つ.
							また,ある時刻で初期波動関数を与えると,それ以降の波動関数は
							一意的に決定される.
				\item 量子力学と古典力学は対応原理で関係付けられる.
			\end{itemize}
			\end{itembox}

		\begin{itembox}[l]{2.3節のまとめ 波動関数の確率解釈とその性質}
			\begin{itemize}
				\item 波動関数の絶対値の2乗は,その場所での粒子の存在確率密度を与える.
							波動関数は規格化されなければならない.
				\item 二重スリットの干渉縞は波動関数を使って説明することができる.
			\end{itemize}
			\end{itembox}

		\begin{itembox}[l]{2.4節のまとめ エネルギーが保存される系のシュレディンガー方程式}
			\begin{itemize}
				\item エネルギーが保存される系では,
							波動関数を時間と空間座標に依存する2つの部分に分離できる.
							$$ \frac{d}{dt}f(t) = -\frac{i}{\hbar}Ef(t), \
									\left[-\frac{\hbar ^2}{2m}\frac{d^2}{dx^2} + V(x)\right]\psi(x) = E\psi(x)$$
				\item 分離後の空間座標のみに依存する部分は,定常状態の波動関数を与える.
			\end{itemize}
			\end{itembox}

		\begin{itembox}[l]{2.5節のまとめ 物理量と演算子}
			\begin{itemize}
				\item 物理量は粒子の状態を表す波動関数に作用する演算子で表される.
							例えば,ハミルトニアン$\hat{H}$,運動量演算子$\hat{p}$,位置の演算子$\hat{x}$
							などがある.
				\item 物理量$A$の期待値は$\average{A} = \int dx\psi^*(x) \hat{A}\psi(x)$
							で定義される.これは測定値に対応するから実数である.このことから,
							物理量を表す演算子はエルミート演算子($\hat{A} = \hat{A}^{\dagger}$)
							でなければならない.
				\item 波動関数が物理量を表す演算子の固有値方程式を満たすとき,その状態での物理量の測定値は
							固有値で与えられる.平面波の解は,運動量演算子と自由粒子のハミルトニアンの同時
							固有関数である.一般に,2つの物理量を表す演算子が交換する場合,それらの物理量に関する
							同時固有関数が存在する.一方,2つの演算子が交換しない場合は,両者の物理量を同時に
							確定することはできない.
				\item 物理量$A$の不確かさは,$A$の期待値と$A^2$の期待値を使ってその標準偏差$\Delta A =
							\sqrt{\average{A^2} - \average{A}^2}$で与えられる.一般に,2つの演算子の不確かさ
							の間には$$(\Delta A)(\Delta B) \geq \frac{1}{2}|\average{[\hat{A},\hat{B}]}|$$なる不等式が成り立つ.これをハイゼンベルクの不確定性関係という.
				\item 交換しない演算子の例として$[\hat{x},\hat{p}] = i\hbar$がある.したがって,
							両者の不確かさの間には$(\Delta x)(\Delta p) \geq \hbar/2$が成り立つ.
			\end{itemize}
			\end{itembox}

		\begin{itembox}[l]{3.1節のまとめ 階段状ポテンシャルによる粒子の散乱}
			\begin{itemize}
				\item シュレーディンガー方程式を解く際には,物理的な条件としての境界条件,接続条件を課す.
							ポテンシャルに無限大の飛びがない場合には,波動関数とその微分は連続でなければならない.
							無限大の飛びがある場合には,その場所で波動関数の微分は連続とならなくてもよい.
				\item 確率の保存を表す連続の方程式が成り立つ.
							確率の流れを使ってポテンシャルによる粒子の反射率や透過率が計算できる.
				\item 古典力学では運動が許されない領域でも,ミクロの世界では粒子の運動が許される場合がある.
							トンネル効果はその代表例である.
			\end{itemize}
			\end{itembox}

		\begin{itembox}[l]{3.2節のまとめ ディラックのデルタ関数}
			\begin{itemize}
				\item デルタ関数は,テスト関数や関数列を用いて定義される.
				\item 階段関数の微分はデルタ関数を与える.
				\item 平面波はデルタ関数で規格化する:
							$$\int_{-\infty}^\infty {\psi^*_{k'}}(x)\psi_k(x)dx = \delta(k' - k)$$
			\end{itemize}
			\end{itembox}

		\begin{itembox}[l]{3.3節のまとめ 井戸型ポテンシャルに束縛された粒子}
			\begin{itemize}
				\item ポテンシャルの中に束縛された粒子のエネルギースペクトルは離散的である.
							このような状態を”量子化される”という.量子化されたエネルギーや波動関数は
							量子数で区別される.波動関数は規格直交系をなす:
							$$\int_{-\infty}^\infty {\psi^*_{m}}(x)\psi_n(x)dx = \delta_{nm}$$
				\item 基底状態にある粒子のエネルギーを零点エネルギーとよぶ.
							基底状態においても粒子は静止できない.
				\item パリティーとは偶奇性のことである.系が適切に設定された座標軸に対して左右対称
							であれば,粒子の波動関数やその他の性質は系のその対称性を反映する.
			\end{itemize}
			\end{itembox}

		\begin{itembox}[l]{3.4節のまとめ 調和振動子型ポテンシャルに束縛された粒子}
			\begin{itemize}
				\item 級数展開や代数的手法を用いてシュレーディンガー方程式を解くことができる.無限遠方で
							波動関数がゼロとなる境界条件から,量子化されたエネルギースペクトルを得ることができる.
							波動関数はエルミート多項式を含む形で表すことができる.
				\item 代数的手法では,演算子$\hat{a},\hat{a}^\dagger, \hat{n}$が重要な役目を果たす.
							第14章ではこのような考え方を使って場の量子論を展開していく.
				\item 調和振動子型ポテンシャルにおける波動関数を使って,ハイゼンベルクの不確定性関係を具体的に
							確認することができる.基底状態では,位置と運動量の不確定性関係が最小となる.
			\end{itemize}
			\end{itembox}
\section{2章 ミクロの世界の力学の仕組み}
	\subsection{1次元の弦の波動方程式}
		位置$x$と時間$t$に関する弦の振動$y(x,t)$は,力学的考察により以下の偏微分方程式を満たす.
		(弦の波動方程式)
			\begin{equation}
				\frac{\pt^2 y(x,t)}{\pt t^2} = v^2 \frac{\pt^2 y(x,t)}{\pt x^2} \label{eq2.1}
			\end{equation}
		この波動方程式の解は2つ考えられる.
		\begin{itemize}
			\item $y = A\sin{(\pm kx - \omega t)}$,$y = A \cos{(\pm kx - \omega t)}$
			\item $y = A\exp{[i(\pm kx - \omega t)]}$
		\end{itemize}
		\textcolor{blue}{
		・ここで$k = 2\pi/\lambda$[rad/m]である.これは波数と呼ばれ,
		「1mあたりの位相(1メートルで何rad進むか)」を示す.周波数$\omega = 2\pi/t$[rad/t]
		が「1sあたりの位相」であることと対照的である.\\
		・$\pm kx$の符号はそれぞれ波が$\pm x$方向に進むことを示す.時刻$t$の方は+の
		1方向のみに進むので$\pm$とはならない.(数学的にはどちらを$\pm$にしてもよい.)
		}\\
		考えられる解($\sin,\cos,\exp$)を式(\ref{eq2.1})に代入すると,
		端数と周波数について以下の関係がわかる.
		\begin{equation}
			\omega ^2 = v^2 k^2( \therefore \omega = vk) \label{eq2.2}
		\end{equation}
		これを(この波動方程式についての){\bf 分散関係}といい,振動について時間と空間の関係を示す.
		また,$v$を位相速度という.
		\begin{itembox}[l]{2.1節のまとめ 1次元の弦の波動方程式}
			\begin{itemize}
				\item 古典的な弦を伝わる波を記述する波動方程式は,三角関数
							や指数関数で表される進行波解をもつ.
				\item 波数と角周波数の関係を分散関係という.
							進行波解で位相が一定の所の進む速度を,位相速度という.
			\end{itemize}
			\end{itembox}
	\subsection{シュレーディンガー方程式}
		シュレーディンガー方程式を発見的に導く.(古典力学の波動方程式からの類推で導く)
		(束縛されない自由な)電子を例にする.電子のエネルギーを,粒子的な記述,波動的な記述の
		二通りで表し,ドブロイ波のスケールでの分散関係を求める,2章1節の分散関係および波動方程式
		の関係から逆算して,ドブロイ波のスケール(ミクロ)での波動方程式の形を類推する.
		まず,粒子的な記述として$$ E = \frac{p^2}{2m}$$
		と書ける.次にドブロイ波の表記として,$$ E = \hbar \omega $$
		と書ける.また,その運動量$p = \hbar k$より($k$は波数)
		\begin{equation}
			\hbar \omega = \frac{\hbar ^2 k^2}{2m} \label{eq2.3}
		\end{equation}
		が得られる.\textcolor{blue}{(式(\ref{eq2.1}),(\ref{eq2.2})の関係を見ながら見て欲しい)}
		式(\ref{eq2.3})の左辺は時間について,右辺は位置についての項であるからこれもまた
		波動方程式から導かれる{\bf 分散関係}であると考える.
		\par 前節の古典的考え方では,式(\ref{eq2.1})
		$$\frac{\pt^2}{\pt t^2} \textcolor{blue}{(時間に関係)}
		= \frac{\pt^2}{\pt x^2}\textcolor{blue}{(位置に関係)}$$
		から式(\ref{eq2.2})
		$$\omega ^2 \textcolor{blue}{(時間に関係)} = k^2 \textcolor{blue}{(位置に関係)}$$
		が導かれた.ここから類推して,
		逆に式(\ref{eq2.3})
		$$\omega \textcolor{blue}{(時間に関係)} = k^2 \textcolor{blue}{(位置に関係)}$$
		は$$\frac{\pt}{\pt t} \textcolor{blue}{(時間に関係)}
				= \frac{\pt^2}{\pt x^2}\textcolor{blue}{(位置に関係)}$$である偏微分方程式
		(波動方程式)から導き出されるのではないかと考える.

\end{document}
